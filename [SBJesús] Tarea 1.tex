\documentclass[11pt]{article}
    \title{\textbf{Task 1 - TALF}}
    \author{Jesús Santos Barba}
    \date{\today}
    
    \addtolength{\topmargin}{-3cm}
    \addtolength{\textheight}{3cm}
    \pagenumbering{arabic}
\begin{document}

\maketitle
\thispagestyle{empty}

\section{Relation in Mathematics}
In this task I am presenting the exercise in the amazing \LaTeX \ editor.\\
But... should not we know first firmly what is a Relation?\\
In mathematics, a binary relation is a general concept that defines some relation between the elements of two sets. It is a generalization of the more commonly understood idea of a mathematical function, but with fewer restrictions.\\
In order to solve the product of a relation we need to compute the Boolean Product n-times multiplied by itself. Of course the matrix used to calculate the power should be the base matrix of the initial relation, that operated with the resulting matrix until you reach to that n-th matrix.


\section{Development of the Power Relation}
The way to go is applying the steps described above. As a result of that, we get the next results:
\begin{description}
\addtolength{\itemindent}{0.80cm}
\itemsep0em 
\item[$R$] \   = \{(1,1), (1,2), (2,3), (3,4)\}
\item[$R^{2}$] = \{(1,1), (1,2), (1,3), (2,4)\}
\item[$R^{3}$] = \{(1,1), (1,2), (1,3), (1,4)\}

\end{description}
Refer to the \emph{Git repository}\footnote{https://github.com/alexandervdm/gummi/wiki/Getting-Involved} that I leave on the end of the document for more.

\end{document}

