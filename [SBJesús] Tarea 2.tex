\documentclass[11pt]{article}
\usepackage{graphicx}
    \title{\textbf{Task 2 - TALF}}
    \author{Jesús Santos Barba}
    \date{\today}
    
    \addtolength{\topmargin}{-3cm}
    \addtolength{\textheight}{3cm}
    \pagenumbering{arabic}
\begin{document}

\maketitle
\thispagestyle{empty}

\section{Specification of the Automaton}

A deterministic finite automaton (DFA) is a 5-tuple $(K, \Sigma, \delta,s, F)$, where
\begin{enumerate}
\item $K$ is a non-empty set of states
\item $\Sigma$ is an alphabet
\item $s \in K$ is the initial state
\item $F \subseteq K$ is a set of final states
\item $\delta: K \times \Sigma \to K$ is the transition function 
\end{enumerate}

\section{Automaton in JFLAP}

In the first activity we are asked to design a Deterministic Finite Automata (DFA) which accepts the strings formed only by a's. That being done in JFLAP.
\begin{center}
	\includegraphics[height=7.5cm]{automata-uwu1.png}
\end{center}

\footnote{https://github.com/alexandervdm/gummi/wiki/Getting-Involved}

\newpage
\section{Automaton in Octave}
A second requirement of the task is to design the automaton in question this time in a JSON file, such that it can be executed on by the Octave environment. The definition looks as follows:

\begin{description}
\addtolength{\itemindent}{0.80cm}
\itemsep0em 
\item["name"] : "activity1" 
\item["representation"] : \{
	\subitem "K" : ["q0", "q1", "q2"],
	\subitem "A" : ["a", "b"],
	\subitem "s" : ["q0"],
	\subitem "F" : ["q1"],
	\subitem "t" : [
	
\setlength\parindent{1.3cm}
			
			\subitem	["q0", "a", "q1"],
			\subitem	["q0", "b", "q2"],
			\subitem	["q1", "a", "q1"],
			\subitem	["q1", "b", "q2"],
			\subitem	["q2", "a", "q2"],
			\subitem	["q2", "b, "q2"]					
					\ ]
	\\	
\}

\end{description}

\setlength\parindent{0cm}

Refer to the \emph{Git repository}\footnote{https://github.com/alexandervdm/gummi/wiki/Getting-Involved} that I leave on the end of the document for more.

\end{document}