\documentclass[11pt]{article}
\usepackage{graphicx}
    \title{\textbf{Task 3 - TALF}}
    \author{Jesús Santos Barba}
    \date{\today}
    
    \addtolength{\topmargin}{-3cm}
    \addtolength{\textheight}{3cm}
    \pagenumbering{arabic}
\begin{document}

\maketitle
\thispagestyle{empty}

\section{Definition of a Turing-Machine}

A TM is a 5-tuple $(K, q_0, \Sigma, \gamma, \delta)$, where
\begin{enumerate}
	\item $K$ is a non-empty, finite set of states,
    \item $q_0 \in K$ is the initial state,
    \item $\Sigma$ is an alphabet, and if $\Sigma_{R}=\{a_0, l, r, h\}$ are reserved words, then $\Sigma \cap \Sigma_{R} = \emptyset$,
    \item $\gamma: K \times \Sigma_{S} \to \Sigma_{I}$ is the instruction function,
    \item $\delta: K \times \Sigma_{S}\to K$ is the transition function 
  \end{enumerate}
where $\Sigma_{S} = \Sigma \cup \{a_0\}$, $\Sigma_{I} = \Sigma \cup \Sigma_R$ and $a_0$ is often referred as blank symbol.\\


\section{Turing-Machine in JFLAP}

As a first task we are asked to design a Turing-Machine based on the solution of the problem 3.4 in the program JFLAP, which is a tool for designing and testing automatons.
\\\\
The solution would look like:
\begin{center}
	\includegraphics[height=6.223cm]{TM-yisus.jpeg} 
\end{center}


\section{Recursive Function}
A second requirement of the task is to define a recursive function for the computation of adding three values.
\\\\
\textbf{ $>$ add(x, y, z) = $< <\pi_1^{1} | \sigma(\pi_3^{3})> | \sigma(\pi_4^{4}) >$ }

\begin{center}
	\includegraphics[height=6.223cm]{rec.jpeg} 
\end{center}

\section{While Program}
Lastly, we are asked to implement a While program that computes the sum of three values.

\begin{description}
\addtolength{\itemindent}{0.80cm}
\itemsep0em 
\item[sum3 = (3, s)] 
\item[s] : \{
	\subitem x4 := 0; 
	\subitem while x3 $\neq$ 0 do
	
\setlength\parindent{1.3cm}
			
			\subitem	x4 := x4 + 1;
			\subitem	x3 := x3 - 1;
					
\setlength\parindent{0.00cm}									
	\subitem od
	\subitem while x2 $\neq$ 0 do

\setlength\parindent{1.3cm}

		\subitem x4 := x4 + 1;
		\subitem x2 := x2 - 1;

\setlength\parindent{0.00cm}
	\subitem od
	\subitem x1 := x4;

\}
\\
\end{description}

\setlength\parindent{0cm}

Refer to the \emph{Git repository}\footnote{https://github.com/black-metroid/talf\_uwu} that I leave on the end of the document for more.

\end{document}